%!TEX TS-program = xelatex
%!TEX encoding = UTF-8 Unicode
% Awesome CV LaTeX Template for CV/Resume
%
% This template has been downloaded from:
% https://github.com/posquit0/Awesome-CV
%
% Author:
% Claud D. Park <posquit0.bj@gmail.com>
% http://www.posquit0.com
%
% Template license:
% CC BY-SA 4.0 (https://creativecommons.org/licenses/by-sa/4.0/)
%


%-------------------------------------------------------------------------------
% CONFIGURATIONS
%-------------------------------------------------------------------------------
% A4 paper size by default, use 'letterpaper' for US letter
\documentclass[12pt, a4paper]{awesome-cv}

% Configure page margins with geometry
\geometry{left=1.4cm, top=.8cm, right=1.4cm, bottom=1.8cm, footskip=.5cm}

% Color for highlights
% Awesome Colors: awesome-emerald, awesome-skyblue, awesome-red, awesome-pink, awesome-orange
%                 awesome-nephritis, awesome-concrete, awesome-darknight
\colorlet{awesome}{awesome-red}
% Uncomment if you would like to specify your own color
% \definecolor{awesome}{HTML}{3E6D9C}

% Colors for text
% Uncomment if you would like to specify your own color
% \definecolor{darktext}{HTML}{414141}
% \definecolor{text}{HTML}{333333}
% \definecolor{graytext}{HTML}{5D5D5D}
% \definecolor{lighttext}{HTML}{999999}
% \definecolor{sectiondivider}{HTML}{5D5D5D}

% Set false if you don't want to highlight section with awesome color
\setbool{acvSectionColorHighlight}{false}

% If you would like to change the social information separator from a pipe (|) to something else
\renewcommand{\acvHeaderSocialSep}{\quad\textbar\quad}

\usepackage[backend=biber,defernumbers=true,style=numeric,sorting=ydnt,maxbibnames=3]{biblatex}
\addbibresource{references.bib}

% Acronyms
% --------
\usepackage{acronym} % \ac[p], \acl[p], \acs[p], \acf[p]
\acrodef{CRDT}[CRDT]{Conflict-free Replicated Data Type}
\acrodefplural{CRDT}[CRDTs]{Conflict-free Replicated Data Types}

% Commands
% --------
\newcommand{\customlink}[2]{
  \href{#2}{\textbf{#1} (\emph{#2})}
}


%-------------------------------------------------------------------------------
%	PERSONAL INFORMATION
%	Comment any of the lines below if they are not required
%-------------------------------------------------------------------------------
% Available options: circle|rectangle,edge/noedge,left/right
% \photo[rectangle,edge,right]{./examples/profile}
\name{Matthieu}{Nicolas}
\position{Docteur-Ingénieur en Informatique {\enskip\cdotp\enskip\enskip} Spécialité Ingénierie Logicielle}
%\address{}

\mobile{(+33) 6 75 98 34 40}
\email{nicolasmatthieu57@gmail.com}
%\dateofbirth{January 1st, 1970}
%\homepage{www.posquit0.com}
\github{MatthieuNICOLAS}
%\linkedin{posquit0}
% \gitlab{gitlab-id}
% \stackoverflow{SO-id}{SO-name}
% \twitter{@twit}
% \skype{skype-id}
% \reddit{reddit-id}
% \medium{madium-id}
% \kaggle{kaggle-id}
% \googlescholar{googlescholar-id}{name-to-display}
%% \firstname and \lastname will be used
% \googlescholar{googlescholar-id}{}
% \extrainfo{extra information}

% \quote{``Be the change that you want to see in the world."}


%-------------------------------------------------------------------------------
\begin{document}

% Print the header with above personal information
% Give optional argument to change alignment(C: center, L: left, R: right)
\makecvheader[C]

% Print the footer with 3 arguments(<left>, <center>, <right>)
% Leave any of these blank if they are not needed
\makecvfooter{}{}{}
  % {\today}{Matthieu Nicolas~~~·~~~CV}{\thepage}


%-------------------------------------------------------------------------------
%	CV/RESUME CONTENT
%	Each section is imported separately, open each file in turn to modify content
%-------------------------------------------------------------------------------
\vspace{-1em}
% %-------------------------------------------------------------------------------
%	SECTION TITLE
%-------------------------------------------------------------------------------
\cvsection{Summary}


%-------------------------------------------------------------------------------
%	CONTENT
%-------------------------------------------------------------------------------
\begin{cvparagraph}

%---------------------------------------------------------
Site Reliability Engineer at fintech company KarrotPay. Have led growth at infrastructure departments in two fintech companies as lead engineer and founding member. 12+ years of diverse software engineering experience with specialties in software architecture design, infrastructure operation, backend development, and security engineering.

Love to contribute to open sources and tech communities by sharing knowledge and experience. Prefers a command line interface environment as a big fan of Vim, Linux, and macOS. Always trying to customize to find the most optimal environment. Interested in devising a better problem-solving method for challenging tasks, and learning new technologies and tools.
\end{cvparagraph}

\cvsection{Déroulement de carrière}

\begin{cventries}

\cventry
    {Ingénieur Recherche \& Développement}
    {Inria, IRISA, équipe Magellan}
    {Rennes}
    {Janvier 2024 - Aujourd'hui}
    {}

\vspace{-1.5em}
\begin{cvparagraph}
    \textbf{SmartObs :} Une plateforme de data monitoring environnementale

    \medskip
    \begin{cvitems} % Description(s) of tasks/responsibilities
        \item Mise à jour de la description et configuration automatique de la plateforme Fog \customlink{LivingFog}{http://www.fogguru.eu/livingfog/}, servant de base au projet.
        \item Intégration et uniformisation de sources de données issues de l'observatoire environnemental de la rivière Kali Gandaki.
        \item Mise en place d'alertes en cas d'interruption des flux de données via les outils \emph{Prometheus} et \emph{Alert manager}.
    \end{cvitems}

    \medskip
    \subentrytitlestyle{Compétences :} Ingénierie logicielle, Fog Computing, Orchestration de conteneurs, Infrastructure as Code.
\end{cvparagraph}

\vspace{1em}
\cventry
    {Doctorant}
    {Université de Lorraine, Loria, équipe Coast}
    {Nancy}
    {Octobre 2017 - Décembre 2022}
    {}

\vspace{-1.5em}
\begin{cvparagraph}
    \textbf{Thèse :} (Ré)Identification efficace dans les types de données répliquées sans conflit (CRDTs)

    \medskip
    \begin{cvitems} % Description(s) of tasks/responsibilities
        \item Étude des types de données répliquées sans conflits (CRDTs), notamment des CRDTs pour le type Séquence et de leurs utilisations dans les systèmes pair-à-pair dynamiques.
        \item Conception d'un nouveau CRDT pour le type Séquence, \emph{RenamableLogootSplit}, incorporant  un mécanisme de renommage pour réduire périodiquement ses métadonnées.
        \item Implémentation de \emph{RenamableLogootSplit} au sein de \customlink{MUTE}{https://mute.loria.fr}, un éditeur de texte temps collaboratif temps réel web pair-à-pair développé au sein de l'équipe de recherche.
        \item Validation expérimentale \emph{RenamableLogootSplit} et de ses performances par le biais de simulations.
        \item Présentation des résultats de recherche obtenus lors du workshop \emph{Principles and Practice of Consistency for Distributed Data} (PaPoC) en 2020 puis dans le journal scientifique \emph{IEEE Transactions on Parallel and Distributed Systems} (TPDS) en 2022.
    \end{cvitems}

    \bigskip
    \subentrytitlestyle{Publications :}
    \begin{description}[labelindent=1.6em,itemsep=-0.3em]
        \item \fullcite{2022-rls-tpds-nicolas}
        \item \fullcite{nicolas:hal-02526724}
    \end{description}

    \medskip
    \subentrytitlestyle{Compétences :} Systèmes distribués, Algorithmie distribuée, Ingénierie logicielle, Programmation.
\end{cvparagraph}

\vspace{1em}
\cventry
    {Ingénieur Recherche \& Développement} % Job title
    {Inria, Loria, équipe Coast} % Organization
    {Nancy} % Location
    {Octobre 2014 - Septembre 2017} % Date(s)
    {}

\vspace{-1.5em}
\begin{cvparagraph}
    \textbf{MUTE :} Un éditeur de texte collaboratif temps réel web pair-à-pair (\href{https://mute.loria.fr}{\emph{https://mute.loria.fr}})

    \medskip
    \begin{cvitems} % Description(s) of tasks/responsibilities
        \item Conception de l'architecture système et logicielle de l'application.
        \item Implémentation de \emph{LogootSplit}, un CRDT pour le type Séquence.
        \item Intégration de \emph{LogootSplit} avec l'éditeur de texte.
        \item Implémentation d'un système d'anti-entropie pour détecter et ré-échanger les modifications perdues.
    \end{cvitems}

    \medskip
    \subentrytitlestyle{Compétences :} Ingénierie logicielle, Programmation, Systèmes distribués.
\end{cvparagraph}

\begin{cvparagraph}
    \textbf{PLM :} Un environnement d'apprentissage de la programmation (\href{http://people.irisa.fr/Martin.Quinson/Teaching/PLM/}{\emph{http://people.irisa.fr/Martin.Quinson/Teaching/PLM/}})

    \medskip
    \begin{cvitems} % Description(s) of tasks/responsibilities
        \item Implémentation et tests d'un mécanisme de capture des traces d'utilisation du logiciel.
        \item Webification du client lourd existant : conception et mise en place d'une architecture orientée services.
        \item Isolation de l'exécution du code des apprenants dans un service dédié.
        \item Déploiement et supervision de l'application.
    \end{cvitems}

    \medskip
    \subentrytitlestyle{Compétences :} Ingénierie logicielle, Programmation.
\end{cvparagraph}

\end{cventries}

\cvsection{Diplômes}

\begin{cventries}

\cventry
  {Doctorat en Informatique} % Job title
  {Université de Lorraine} % Organization
  {Nancy} % Location
  {2017 - 2022} % Date(s)
  {}

\vspace{-1em}


\cventry
    {Diplôme d'ingénieur TELECOM Nancy, spécialité Ingénierie du Logiciel} % Job title
    {TELECOM Nancy} % Organization
    {Nancy} % Location
    {2011 - 2014} % Date(s)
    {}

\vspace{-1em}

\cventry
  {Diplôme Universitaire de Technologie, spécialité Informatique} % Job title
  {IUT de Metz} % Organization
  {Metz} % Location
  {2009 - 2011} % Date(s)
  {}
\end{cventries}


\pagebreak

\cvsection{Compétences}

\begin{cvskills}
    \cvskill{Programmation}{Python, TypeScript, Angular, HTML/CSS, Java, Scala, SQL.}
    \cvskill{Ingénierie Logicielle}{Architecture système, Architecture logicielle, Gestion de versions, Intégration continue, Conteneurisation, Orchestration.}
    \cvskill{Algorithmie Distribuée}{Mécanismes de résolution de conflits automatiques, Protocoles de consensus, Mécanismes d'anti-entropie.}
    \cvskill{Systèmes Distribués}{CRDTs, Réplication de données, Systèmes pair-à-pair.}
    \cvskill{Langues}{Français (maternelle), Anglais (courant).}
\end{cvskills}

\cvsection{Enseignement}

\begin{cvparagraph}
    En parallèle de mon doctorat, j'ai effectué diverses missions d'enseignements au sein des différentes composantes de l'Université de Lorraine.
    Ainsi, j'ai effectué un service de demi-ATER puis d'ATER complet à Polytech Nancy au cours des années 2020-2022.
    Auparavant, j'ai assuré une charge d'enseignement en tant que DCCE à l'IUT Nancy Charlemagne (2017-2020) et effectué plusieurs vacations à TELECOM Nancy (2014-2017) ainsi qu'à la Faculté des Sciences et Technologies de Nancy (2016-2017).

    Au total, je comptabilise \textbf{542h équivalent TD} d'enseignement.

    \bigskip

    \begin{cvitems} % Description(s) of tasks/responsibilities
        \item Enseignement à un public allant de Licence 1 (Algorithmique, Conception Orientée Objet,...) à Master 2 (Mise en Production de Programmes).
        \item Réalisation de supports de cours, de sujets d'exercices et d'examens.
        \item Coordination des chargé-es de TDs.
    \end{cvitems}
\end{cvparagraph}

\cvsection{Encadrement}

\begin{cventries}

\cventry
  {Stage DUT}
  {Co-encadrement avec Victorien Elvinger}
  {}
  {Avril - Juillet 2020}
  {
    \begin{cvparagraph}
      Tom Mendez-Porcel : \emph{Implémentation d'un protocole de gestion de groupe au sein d'une application d'édition collaborative}.
    \end{cvparagraph}
  }

\vspace{-1em}
\cventry
  {Stage TELECOM Nancy 2a (eq. Master 1)}
  {Co-encadrement avec Cédric Enclos}
  {}
  {Juin - Août 2019}
  {
    \begin{cvparagraph}
      Ishara Chan-Tung : \emph{Intégration d'un agent de messages basé sur des journaux au sein d'une application d'édition collaborative}.
    \end{cvparagraph}
}

\vspace{-1em}
\cventry
  {Projet d'initiation à la recherche TELECOM Nancy 2a}
  {Co-encadrement avec Quentin Laporte-Chabasse}
  {}
  {Janvier - Mai 2017}
  {
    \begin{cvparagraph}
      Pierre Maeckereel, Yannick Philippe : \emph{Simulation du comportement de collaborateurs dans une session d'edition collaborative}.
    \end{cvparagraph}
  }
\end{cventries}

\cvsection{Centres d'intérêts}

\begin{cvskills}
    \cvskill{Jeux Vidéo}{Tombé dedans dès l'enfance, je m'intéresse désormais aux créations de la scène indé (\emph{Outer Wilds}, \emph{CrossCode}, \emph{Hollow Knight}).}
    \cvskill{Jeux de Plateau}{Il y a 10 ans, mes ami-es m'ont appris que le monde ne se limitait pas au \emph{Monopoly} (\emph{Ghost Stories, The Loop, Sherlock Holmes}).}
    \cvskill{Lecture}{Ayant grandi en suivant les aventures d'un certain apprenti sorcier, j'ai toujours un livre en cours (\emph{Pratchett}, \emph{Hobb}, \emph{Christie}, \emph{King}).}
    \cvskill{Natation}{Après plusieurs années de pratique en club, la natation reste aujourd'hui un de mes meilleurs moyens de me relaxer.}
\end{cvskills}

% %-------------------------------------------------------------------------------
%	SECTION TITLE
%-------------------------------------------------------------------------------
\cvsection{Honors \& Awards}


%-------------------------------------------------------------------------------
%	SUBSECTION TITLE
%-------------------------------------------------------------------------------
\cvsubsection{International Awards}


%-------------------------------------------------------------------------------
%	CONTENT
%-------------------------------------------------------------------------------
\begin{cvhonors}

%---------------------------------------------------------
  \cvhonor
    {2nd Place} % Award
    {AWS ASEAN AI/ML GameDay} % Event
    {Online} % Location
    {2021} % Date(s)

%---------------------------------------------------------
  \cvhonor
    {Finalist} % Award
    {DEFCON 28th CTF Hacking Competition World Final} % Event
    {Las Vegas, U.S.A} % Location
    {2020} % Date(s)

%---------------------------------------------------------
  \cvhonor
    {Finalist} % Award
    {DEFCON 26th CTF Hacking Competition World Final} % Event
    {Las Vegas, U.S.A} % Location
    {2018} % Date(s)

%---------------------------------------------------------
  \cvhonor
    {Finalist} % Award
    {DEFCON 25th CTF Hacking Competition World Final} % Event
    {Las Vegas, U.S.A} % Location
    {2017} % Date(s)

%---------------------------------------------------------
  \cvhonor
    {Finalist} % Award
    {DEFCON 22nd CTF Hacking Competition World Final} % Event
    {Las Vegas, U.S.A} % Location
    {2014} % Date(s)

%---------------------------------------------------------
  \cvhonor
    {Finalist} % Award
    {DEFCON 21st CTF Hacking Competition World Final} % Event
    {Las Vegas, U.S.A} % Location
    {2013} % Date(s)

%---------------------------------------------------------
  \cvhonor
    {Finalist} % Award
    {DEFCON 19th CTF Hacking Competition World Final} % Event
    {Las Vegas, U.S.A} % Location
    {2011} % Date(s)

%---------------------------------------------------------
\end{cvhonors}


%-------------------------------------------------------------------------------
%	SUBSECTION TITLE
%-------------------------------------------------------------------------------
\cvsubsection{Domestic Awards}


%-------------------------------------------------------------------------------
%	CONTENT
%-------------------------------------------------------------------------------
\begin{cvhonors}

%---------------------------------------------------------
  \cvhonor
    {2nd Place} % Award
    {AWS Korea GameDay} % Event
    {Seoul, S.Korea} % Location
    {2021} % Date(s)

%---------------------------------------------------------
  \cvhonor
    {3rd Place} % Award
    {WITHCON Hacking Competition Final} % Event
    {Seoul, S.Korea} % Location
    {2015} % Date(s)

%---------------------------------------------------------
  \cvhonor
    {Silver Prize} % Award
    {KISA HDCON Hacking Competition Final} % Event
    {Seoul, S.Korea} % Location
    {2017} % Date(s)

%---------------------------------------------------------
  \cvhonor
    {Silver Prize} % Award
    {KISA HDCON Hacking Competition Final} % Event
    {Seoul, S.Korea} % Location
    {2013} % Date(s)

%---------------------------------------------------------
\end{cvhonors}

%-------------------------------------------------------------------------------
%	SUBSECTION TITLE
%-------------------------------------------------------------------------------
\cvsubsection{Community}


%-------------------------------------------------------------------------------
%	CONTENT
%-------------------------------------------------------------------------------
\begin{cvhonors}

%---------------------------------------------------------
  \cvhonor
    {AWS Community Builder (Container)} % Award
    {Amazon Web Services (AWS)} % Event
    {} % Location
    {2022} % Date(s)

%---------------------------------------------------------
  \cvhonor
    {HashiCorp Ambassador} % Award
    {HashiCorp} % Event
    {} % Location
    {2022} % Date(s)

%---------------------------------------------------------
\end{cvhonors}

% %-------------------------------------------------------------------------------
%	SECTION TITLE
%-------------------------------------------------------------------------------
\cvsection{Certificates}


%-------------------------------------------------------------------------------
%	CONTENT
%-------------------------------------------------------------------------------
\begin{cvhonors}

%---------------------------------------------------------
  \cvhonor
    {AWS Certified Security - Specialty} % Name
    {Amazon Web Services (AWS)} % Issuer
    {} % Credential ID
    {2022} % Date(s)

%---------------------------------------------------------
  \cvhonor
    {AWS Certified Solutions Architect – Professional} % Name
    {Amazon Web Services (AWS)} % Issuer
    {} % Credential ID
    {2022} % Date(s)

%---------------------------------------------------------
  \cvhonor
    {AWS Certified Solutions Architect – Associate} % Name
    {Amazon Web Services (AWS)} % Issuer
    {} % Credential ID
    {2019} % Date(s)

%---------------------------------------------------------
  \cvhonor
    {AWS Certified SysOps Administrator – Associate} % Name
    {Amazon Web Services (AWS)} % Issuer
    {} % Credential ID
    {2021} % Date(s)

%---------------------------------------------------------
  \cvhonor
    {Certified Kubernetes Application Developer (CKAD)} % Name
    {The Linux Foundation} % Issuer
    {} % Credential ID
    {2020} % Date(s)

%---------------------------------------------------------
  \cvhonor
    {HashiCorp Certified: Terraform Associate} % Name
    {HashiCorp} % Issuer
    {} % Credential ID
    {2020} % Date(s)

%---------------------------------------------------------
\end{cvhonors}

% %-------------------------------------------------------------------------------
%	SECTION TITLE
%-------------------------------------------------------------------------------
\cvsection{Presentation}


%-------------------------------------------------------------------------------
%	CONTENT
%-------------------------------------------------------------------------------
\begin{cventries}

%---------------------------------------------------------
  \cventry
    {Presenter for <Hosting Web Application for Free utilizing GitHub, Netlify and CloudFlare>} % Role
    {DevFest Seoul by Google Developer Group Korea} % Event
    {Seoul, S.Korea} % Location
    {Nov. 2017} % Date(s)
    {
      \begin{cvitems} % Description(s)
        \item {Introduced the history of web technology and the JAM stack which is for the modern web application development.}
        \item {Introduced how to freely host the web application with high performance utilizing global CDN services.}
      \end{cvitems}
    }

%---------------------------------------------------------
  \cventry
    {Presenter for <DEFCON 20th : The way to go to Las Vegas>} % Role
    {6th CodeEngn (Reverse Engineering Conference)} % Event
    {Seoul, S.Korea} % Location
    {Jul. 2012} % Date(s)
    {
      \begin{cvitems} % Description(s)
        \item {Introduced CTF(Capture the Flag) hacking competition and advanced techniques and strategy for CTF}
      \end{cvitems}
    }

%---------------------------------------------------------
\end{cventries}

% %-------------------------------------------------------------------------------
%	SECTION TITLE
%-------------------------------------------------------------------------------
\cvsection{Writing}


%-------------------------------------------------------------------------------
%	CONTENT
%-------------------------------------------------------------------------------
\begin{cventries}

%---------------------------------------------------------
  \cventry
    {Founder \& Writer} % Role
    {A Guide for Developers in Start-up} % Title
    {Facebook Page} % Location
    {Jan. 2015 - PRESENT} % Date(s)
    {
      \begin{cvitems} % Description(s)
        \item {Drafted daily news for developers in Korea about IT technologies, issues about start-up.}
      \end{cvitems}
    }

%---------------------------------------------------------
  \cventry
    {Undergraduate Student Reporter} % Role
    {AhnLab} % Title
    {S.Korea} % Location
    {Oct. 2012 - Jul. 2013} % Date(s)
    {
      \begin{cvitems} % Description(s)
        \item {Drafted reports about IT trends and Security issues on AhnLab Company magazine.}
      \end{cvitems}
    }

%---------------------------------------------------------
\end{cventries}

% %-------------------------------------------------------------------------------
%	SECTION TITLE
%-------------------------------------------------------------------------------
\cvsection{Program Committees}


%-------------------------------------------------------------------------------
%	CONTENT
%-------------------------------------------------------------------------------
\begin{cvhonors}

%---------------------------------------------------------
  \cvhonor
    {Problem Writer} % Position
    {2016 CODEGATE Hacking Competition World Final} % Committee
    {S.Korea} % Location
    {2016} % Date(s)

%---------------------------------------------------------
  \cvhonor
    {Organizer \& Co-director} % Position
    {1st POSTECH Hackathon} % Committee
    {S.Korea} % Location
    {2013} % Date(s)

%---------------------------------------------------------
\end{cvhonors}

% %-------------------------------------------------------------------------------
%	SECTION TITLE
%-------------------------------------------------------------------------------
\cvsection{Extracurricular Activity}


%-------------------------------------------------------------------------------
%	CONTENT
%-------------------------------------------------------------------------------
\begin{cventries}

%---------------------------------------------------------
  \cventry
    {Core Member} % Affiliation/role
    {B10S (B1t 0n the Security, Underground hacker team)} % Organization/group
    {S.Korea} % Location
    {Nov. 2011 - PRESENT} % Date(s)
    {
      \begin{cvitems} % Description(s) of experience/contributions/knowledge
        \item {Gained expertise in penetration testing areas, especially targeted on web application and software.}
        \item {Participated on a lot of hacking competition and won a good award.}
        \item {Held several hacking competitions non-profit, just for fun.}
      \end{cvitems}
    }

%---------------------------------------------------------
  \cventry
    {Member} % Affiliation/role
    {WiseGuys (Hacking \& Security research group)} % Organization/group
    {S.Korea} % Location
    {Jun. 2012 - PRESENT} % Date(s)
    {
      \begin{cvitems} % Description(s) of experience/contributions/knowledge
        \item {Gained expertise in hardware hacking areas from penetration testing on several devices including wireless router, smartphone, CCTV and set-top box.}
        \item {Trained wannabe hacker about hacking technique from basic to advanced and ethics for white hackers by hosting annual Hacking Camp.}
      \end{cvitems}
    }

%---------------------------------------------------------
  \cventry
    {Core Member \& President at 2013} % Affiliation/role
    {PoApper (Developers' Network of POSTECH)} % Organization/group
    {Pohang, S.Korea} % Location
    {Jun. 2010 - Jun. 2017} % Date(s)
    {
      \begin{cvitems} % Description(s) of experience/contributions/knowledge
        \item {Reformed the society focusing on software engineering and building network on and off campus.}
        \item {Proposed various marketing and network activities to raise awareness.}
      \end{cvitems}
    }

%---------------------------------------------------------
  \cventry
    {Member} % Affiliation/role
    {PLUS (Laboratory for UNIX Security in POSTECH)} % Organization/group
    {Pohang, S.Korea} % Location
    {Sep. 2010 - Oct. 2011} % Date(s)
    {
      \begin{cvitems} % Description(s) of experience/contributions/knowledge
        \item {Gained expertise in hacking \& security areas, especially about internal of operating system based on UNIX and several exploit techniques.}
        \item {Participated on several hacking competition and won a good award.}
        \item {Conducted periodic security checks on overall IT system as a member of POSTECH CERT.}
        \item {Conducted penetration testing commissioned by national agency and corporation.}
      \end{cvitems}
    }

%---------------------------------------------------------
  \cventry
    {Member} % Affiliation/role
    {MSSA (Management Strategy Club of POSTECH)} % Organization/group
    {Pohang, S.Korea} % Location
    {Sep. 2013 - Jun. 2017} % Date(s)
    {
      \begin{cvitems} % Description(s) of experience/contributions/knowledge
        \item {Gained knowledge about several business field like Management, Strategy, Financial and marketing from group study.}
        \item {Gained expertise in business strategy areas and inisght for various industry from weekly industry analysis session.}
      \end{cvitems}
    }

%---------------------------------------------------------
\end{cventries}



%-------------------------------------------------------------------------------
\end{document}
