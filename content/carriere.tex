\cvsection{Déroulement de carrière}

\begin{cventries}

\cventry
    {Doctorant}
    {Université de Lorraine, Loria, équipe Coast}
    {Nancy}
    {Octobre 2017 - Décembre 2022}
    {}

\vspace{-1.5em}
\begin{cvparagraph}
    \textbf{Thèse :} (Ré)Identification efficace dans les types de données répliquées sans conflit (CRDTs)

    \medskip
    \begin{cvitems} % Description(s) of tasks/responsibilities
        \item Étude des types de données répliquées sans conflits (CRDTs), notamment des CRDTs pour le type Séquence et de leurs utilisations dans les systèmes pair-à-pair dynamiques.
        \item Conception d'un nouveau CRDT pour le type Séquence, \emph{RenamableLogootSplit}, incorporant  un mécanisme de renommage pour réduire périodiquement ses métadonnées.
        \item Implémentation de \emph{RenamableLogootSplit} au sein de \customlink{MUTE}{https://mutehost.loria.fr}, un éditeur de texte collaboratif web pair-à-pair développé au sein de l'équipe de recherche.
        \item Validation expérimentale \emph{RenamableLogootSplit} et de ses performances par le biais de simulations.
        \item Présentation des résultats de recherche obtenus lors du workshop \emph{Principles and Practice of Consistency for Distributed Data} (PaPoC) en 2020 puis dans le journal scientifique \emph{IEEE Transactions on Parallel and Distributed Systems} (TPDS) en 2022.
    \end{cvitems}

    \bigskip
    \subentrytitlestyle{Publications :}
    \begin{description}[labelindent=1.6em,itemsep=-0.3em]
        \item \fullcite{2022-rls-tpds-nicolas}
        \item \fullcite{nicolas:hal-02526724}
    \end{description}

    \medskip
    \subentrytitlestyle{Compétences :} Systèmes distribués (CRDTs, Réplication de données, Systèmes pair-à-pair), Algorithmie distribuée (Mécanismes de résolution de conflits automatiques, Protocoles de consensus, Mécanismes d'anti-entropie, Protocoles de gestion d'appartenance au groupe), Ingénierie logicielle (Architecture système et logicielle, Intégration continue, Conteneurisation), Programmation web (TypeScript).
\end{cvparagraph}

\begin{cvparagraph}
    \textbf{Enseignement :} IUT Nancy-Charlemagne (DCCE, 2017-2020) puis Polytech Nancy (ATER, 2020-2022)

    \medskip
    \begin{cvitems} % Description(s) of tasks/responsibilities
        \item Total de 542h équivalent TD d'enseignement, à un public allant de Licence 1 (Algorithmique, Conception Orientée Objet,...) à Master 2 (Mise en Production de Programmes).
        \item Réalisation de supports de cours, de sujets d'exercices et d'examens.
        \item Coordination des chargé-es de TDs.
    \end{cvitems}

    \medskip
    \subentrytitlestyle{Compétences :} Algorithmie, Conception Orientée Objet, Programmation Web, Bases de données relationnelles, Tests unitaires.
\end{cvparagraph}

\cventry
    {Ingénieur Recherche \& Développement} % Job title
    {INRIA, équipe Coast} % Organization
    {Nancy} % Location
    {Octobre 2014 - Septembre 2017} % Date(s)
    {}

\vspace{-1.5em}
\begin{cvparagraph}
    \textbf{MUTE :} Un éditeur de texte collaboratif web pair-à-pair (\href{https://mutehost.loria.fr}{\emph{https://mutehost.loria.fr}})

    \medskip
    \begin{cvitems} % Description(s) of tasks/responsibilities
        \item Conception de l'architecture système et logicielle de l'application.
        \item Implémentation de \emph{LogootSplit}, un CRDT pour le type Séquence.
        \item Intégration de \emph{LogootSplit} avec l'éditeur de texte.
        \item Implémentation d'un système d'anti-entropie pour détecter et ré-échanger les modifications perdues.
    \end{cvitems}

    \medskip
    \subentrytitlestyle{Compétences :} Ingénierie logicielle (Architecture système et logicielle, Intégration Continue), Programmation web (TypeScript, Angular, Node.js), CRDTs, Local-First Softwares, Systèmes pair-à-pair.
\end{cvparagraph}

\begin{cvparagraph}
    \textbf{PLM :} Un environnement d'apprentissage de la programmation (\href{http://people.irisa.fr/Martin.Quinson/Teaching/PLM/}{\emph{http://people.irisa.fr/Martin.Quinson/Teaching/PLM/}})

    \medskip
    \begin{cvitems} % Description(s) of tasks/responsibilities
        \item Implémentation et tests d'un mécanisme de capture des traces d'utilisation du logiciel.
        \item Webification du client lourd existant : conception et mise en place d'une architecture orientée services.
        \item Isolation de l'exécution du code des apprenants dans un service dédié.
        \item Déploiement et supervision de l'application.
    \end{cvitems}

    \medskip
    \subentrytitlestyle{Compétences :} Ingénierie logicielle (Architecture système et logicielle, Intégration Continue, Conteneurisation), Programmation (Scala, JavaScript).
\end{cvparagraph}

\end{cventries}
