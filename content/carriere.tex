\cvsection{Déroulement de carrière}

\begin{cventries}

\cventry
    {Ingénieur Recherche \& Développement}
    {Inria, IRISA, équipe Magellan}
    {Rennes}
    {Janvier 2024 - Aujourd'hui}
    {}

\vspace{-1.5em}
\begin{cvparagraph}
    \textbf{SmartObs :} Une plateforme de data monitoring environnementale

    \medskip
    \begin{cvitems} % Description(s) of tasks/responsibilities
        \item Mise à jour de la description et configuration automatique de la plateforme Fog \customlink{LivingFog}{http://www.fogguru.eu/livingfog/}, servant de base au projet.
        \item Intégration et uniformisation de sources de données issues de l'observatoire environnemental de la rivière Kali Gandaki.
        \item Mise en place d'alertes en cas d'interruption des flux de données via les outils \emph{Prometheus} et \emph{Alert manager}.
    \end{cvitems}

    \medskip
    \subentrytitlestyle{Compétences :} Ingénierie logicielle, Fog Computing, Orchestration de conteneurs, Infrastructure as Code.
\end{cvparagraph}

\vspace{1em}
\cventry
    {Doctorant}
    {Université de Lorraine, Loria, équipe Coast}
    {Nancy}
    {Octobre 2017 - Décembre 2022}
    {}

\vspace{-1.5em}
\begin{cvparagraph}
    \textbf{Thèse :} (Ré)Identification efficace dans les types de données répliquées sans conflit (CRDTs)

    \medskip
    \begin{cvitems} % Description(s) of tasks/responsibilities
        \item Étude des types de données répliquées sans conflits (CRDTs), notamment des CRDTs pour le type Séquence et de leurs utilisations dans les systèmes pair-à-pair dynamiques.
        \item Conception d'un nouveau CRDT pour le type Séquence, \emph{RenamableLogootSplit}, incorporant  un mécanisme de renommage pour réduire périodiquement ses métadonnées.
        \item Implémentation de \emph{RenamableLogootSplit} au sein de \customlink{MUTE}{https://mute.loria.fr}, un éditeur de texte temps collaboratif temps réel web pair-à-pair développé au sein de l'équipe de recherche.
        \item Validation expérimentale \emph{RenamableLogootSplit} et de ses performances par le biais de simulations.
        \item Présentation des résultats de recherche obtenus lors du workshop \emph{Principles and Practice of Consistency for Distributed Data} (PaPoC) en 2020 puis dans le journal scientifique \emph{IEEE Transactions on Parallel and Distributed Systems} (TPDS) en 2022.
    \end{cvitems}

    \bigskip
    \subentrytitlestyle{Publications :}
    \begin{description}[labelindent=1.6em,itemsep=-0.3em]
        \item \fullcite{2022-rls-tpds-nicolas}
        \item \fullcite{nicolas:hal-02526724}
    \end{description}

    \medskip
    \subentrytitlestyle{Compétences :} Systèmes distribués, Algorithmie distribuée, Ingénierie logicielle, Programmation.
\end{cvparagraph}

\vspace{1em}
\cventry
    {Ingénieur Recherche \& Développement} % Job title
    {Inria, Loria, équipe Coast} % Organization
    {Nancy} % Location
    {Octobre 2014 - Septembre 2017} % Date(s)
    {}

\vspace{-1.5em}
\begin{cvparagraph}
    \textbf{MUTE :} Un éditeur de texte collaboratif temps réel web pair-à-pair (\href{https://mute.loria.fr}{\emph{https://mute.loria.fr}})

    \medskip
    \begin{cvitems} % Description(s) of tasks/responsibilities
        \item Conception de l'architecture système et logicielle de l'application.
        \item Implémentation de \emph{LogootSplit}, un CRDT pour le type Séquence.
        \item Intégration de \emph{LogootSplit} avec l'éditeur de texte.
        \item Implémentation d'un système d'anti-entropie pour détecter et ré-échanger les modifications perdues.
    \end{cvitems}

    \medskip
    \subentrytitlestyle{Compétences :} Ingénierie logicielle, Programmation, Systèmes distribués.
\end{cvparagraph}

\begin{cvparagraph}
    \textbf{PLM :} Un environnement d'apprentissage de la programmation (\href{http://people.irisa.fr/Martin.Quinson/Teaching/PLM/}{\emph{http://people.irisa.fr/Martin.Quinson/Teaching/PLM/}})

    \medskip
    \begin{cvitems} % Description(s) of tasks/responsibilities
        \item Implémentation et tests d'un mécanisme de capture des traces d'utilisation du logiciel.
        \item Webification du client lourd existant : conception et mise en place d'une architecture orientée services.
        \item Isolation de l'exécution du code des apprenants dans un service dédié.
        \item Déploiement et supervision de l'application.
    \end{cvitems}

    \medskip
    \subentrytitlestyle{Compétences :} Ingénierie logicielle, Programmation.
\end{cvparagraph}

\end{cventries}
